\documentclass[a4paper, 10pt]{article}

\usepackage{wrapfig}
\usepackage{listings}
%Math
\usepackage{amsmath}
\usepackage{amsfonts}
\usepackage{amssymb}
\usepackage{amsthm}
\usepackage{ulem}
\usepackage{textcomp}


%PageStyle
\usepackage[german]{babel}
\usepackage{fontenc}
\usepackage{fancyhdr, graphicx}
\usepackage{fullpage}
\usepackage{graphicx}
\usepackage{textcomp}
\usepackage{fancyhdr} %for header/footer
\usepackage{wrapfig}


\newcommand{\Bold}[1]{\textbf{#1}} %Boldface
\newcommand{\Kursiv}[1]{\textit{#1}} %Italic

%Metadata
\title{Betriebssysteme Test 2}
\author{Jan F\"assler}
\date{2. Semester (FS 2012)}


\begin{document}
\maketitle
\newpage
\section{Parallelität - Nebenläufigkeit}
\subsection{Einleitung}
Falls ein Rechner mehrere Prozesse parallel verarbeiten kann (Multi-Tasking), so führt er jeden Prozess als separate Aktivität aus. Diese Prozesse können unabhängig voneinander ablaufen (Nebenläufigkeit), müssen jedoch synchronisiert werden, da bestimmte Ressourcen (insbesondere die CPU) gemeinsam genutzt werden. Stehen mehrere CPUs zur Verfügung, können Prozesse parallel ablaufen, jedoch weiterhin mit Synchronisation und Wartezuständen bei gemeinsam genutzten Ressourcen. Werden Threads innerhalb eines Prozess-Adressraums eingesetzt, gilt diese Aussage sinngemäss auch für jeden Thread innerhalb eines Prozesskontexts.

\subsection{Der Sheduler}
Der Scheduler teilt Prozessen Rechenzeit zu und sequentialisiert damit die nebenläufige Ausführung von Prozessen oder Threads auf einer oder mehreren CPUs. Bei genügender Performance des Systems geschieht dies so rasch, dass für den Benutzer der Eindruck der Parallelität entsteht. Neben der Ressourcenauslastung beeinflusst die Scheduling-Strategie den Grad der Quasi-Parallelität.

\subsection{Pre-emption}
Prozesse können in Unix zu fast jeder Zeit unterbrochen werden:
\begin{itemize}
	\item freiwillig durch Systemaufruf mit Wartezustand oder Abgabe der CPU (sleep, wait)
	\item ungeplant durch den Scheduler auf Basis von Priorisierung und Ressourcenverbrauch oder nach Ablauf der Zeitscheibe
	\item ungeplant durch asynchronen Events (z.B. Interrupts, die durch den gerade laufenden Prozess behandelt werden müssen)
\end{itemize}
Ein Prozess muss seinen Ausführungskontext unter- brechen, abspeichern, zum neuen Kontext wechseln, im neuen Kontext ablaufen, den alten Kontext wiederherstellen und neu starten können (Aufgabe des Kernels). Behandelt der Prozess im Programmcode Ausnahmen selbst (z.B. Signale), muss der Programmierer selbst für die Konsistenz von Variablen und Zuständen sorgen.

\subsection{Synchronisation}
\subsubsection{Das Problem der Synchronisierung}
5 Personen sitzen am Tisch, zwischen den Personen liegen 5 Stäbchen. Zum Essen benötigt jede Person 2 Stäbchen.
\begin{description}
	\item[Warten] Griff in‘s Leere
	\item[Synchronisation] Gleichzeitiger Zugriff
	\item[Deadlock] jede Person nimmt das rechte Stäbchen und wartet auf das linke
	\item[Starvation] alle Personen sollen in sinnvolle Frist essen
\end{description}

\subsubsection{Synchronisation}
Zwischen Kernel und Prozessen:
\begin{itemize}
	\item Signalisierung	
	\item Schlaf-/Wartezustand des Prozesses
	\item Aufwecken \& Scheduling
\end{itemize}
Zwischen Prozessen:
\begin{itemize}
	\item Einseitige Synchronisation
	\item Mehrseitige Synchronisation
	\item Gegenseitiger Ausschluss aus kritischen Abschnitten
\end{itemize}

\subsection{Synchronisationsmittel}

\subsubsection{Benachrichtigung \& Warten}
\begin{itemize}
	\item Prozess wartet aktiv auf das Eintreffen einer Nachricht (busy waiting)
	\item Prozess wartet(schläft) auf das Eintreffen eines Wecksignals (typisiert) – der Signal-Mechanismus von Unix weckt dann (via den Kernel / Scheduler) alle auf dieses Ereignis wartenden Prozesse
\end{itemize}

\subsubsection{Locks/Schlossvariablen}
\begin{itemize}
	\item Im Dateisystem (Lockfile, Lock-Bits im Superblock)
	\item Im Speicher (Lock Bits, gemeinsame Variablen)
	\item Modell: der Prozess prüft zyklisch auf Zustandsänderung des Bits / der Variablen und modifiziert das Bit bzw. die Variable, falls erlaubt.
	\item Problem: meist keine atomare Operation wegen jederzeitiger Unterbrechbarkeit, es kann daher zu Problemen kommen, wenn die Implementation nicht durch unteilbare CPU-Instruktionen unterstützt wird.
\end{itemize}

\subsubsection{Ereigniszähler}
\begin{itemize}
	\item Variante eines Locks, welches durch eine Zählervariable (z.B. in einem Shared Memory Segment) implementiert wird.
	\item Sinnvoll, wenn mehrere Prozesse zu synchronisieren sind, z.B. maximale Anzahl quasiparalleler Leseprozesse auf einer Datenbank wegen Performance-Garantien.
	\item Gleiches Problem der unterbrechbaren, nicht atomaren Operation wie bei Locks, benötigt daher ähnliche Schutzmechanismen.
\end{itemize}

\subsubsection{Petri-Netze}
Modellierung von nebenläufigen Systemen und ihrer Synchronisation
\begin{itemize}
	\item Stellen = Kreis
	\item Transitionen = Rechteck
	\item Marken = Punkt
	\item Schaltregeln = alle vorausgehenden Stellen enthalten mind. 1 Marke, alle nachfolgenden Stellen erhalten eine Marke
\end{itemize}

\subsubsection{Dijkstra Semaphore}
\begin{itemize}
	\item Modul/Kapsel mit geschützter Statusvariablen
	\item Bereitstellung von Operationen für:
		\begin{itemize}
				\item Initialisierung
				\item Eintritt/Signalisierung (P)
				\item Austritt/Freigabe (V)
				\item Deallokation
		\end{itemize}
	\item Typen:
		\begin{itemize}
			\item Binär	
			\item Zähler
			\item Set / Array
		\end{itemize}
	\item Implementierung: atomare CPU-Instruktion oder kurzzeitige Erhöhung des Interrupt-Levels
\end{itemize}

\subsubsection{Barrier}
Die fehlerhafte Programmierung eines wechselseitigen Ausschlusses mit Semaphoren kann zu signifikanten Fehlern führen - neues Sprachkonstrukt MONITOR ohne explizite Programmierung von P und V Operationen – stattdessen Generierung durch den Compiler .
\begin{itemize}
	\item Modul, welches Daten und Methoden/Prozeduren enthält.
	\item Aufruf des Monitor Entry durch beliebig viele Prozesse; Garantie des wechselseitigen Ausschlusses
	\item Prozeduren/Methoden eines Monitors können auf globale Daten zugreifen, die lokalen Daten des Monitors sind aber von aussen nicht zugänglich
	\item Im Innern eines Monitors ist es ggf. nötig, dass eine Aktivität wartet. Eine solche Aktivität wird durch den Monitor aus dem Monitor ausgelagert. Auf diese Weise bleibt der Monitor nicht blockiert und eine andere Aktivität kann in den Monitor eintreten. Falls eine andere Aktivität den Wartenden befreit, so kann diese, sobald der Monitor frei ist, diesen wieder betreten.	
\end{itemize}

\subsubsection{Rendezvous}
\begin{itemize}
	\item Synchronisation von entfernten Prozedurauf- rufen (remote procedure calls)
	\item Der Prozedur-Aufrufer wird blockiert, bis die ent- fernte Prozedur ausgeführt wurde
	\item Der Prozedur-Anbieter bleibt blockiert, bis eine seiner Prozeduren aufgerufen wird.
	\item Operationen:
		\begin{itemize}
			\item Rendenzvous anbieten (Anbieter)
			\item Rendezvous beantragen (Aufrufer) - Warteschlange
			\item Rendezvous annehmen (Anbieter) - erster Aufrufer
			\item Rendezvous ausführen \& Resultat melden (Anbieter)
		\end{itemize}
\end{itemize}

\subsection{Deadlock Erkennung und Vermeidung/Behebung}
Deadlocks treten auf, wenn:
\begin{itemize}
	\item die umstrittenen Ressourcen nur exklusiv nutzbar sind,
	\item die umstrittenen Ressourcen nicht entzogen werden können,
	\item die Belegung von Ressourcen schon möglich ist, auch wenn auf die Zuweisung weiterer Ressourcen gewartet werden muss,
	\item eine zyklische Kette von Prozessen auftritt, in der jeder Prozess mindestens eine Ressource besitzt, die der nächste Prozess in der Kette benötigt.
\end{itemize}
Gegenmassnahmen:
\begin{itemize}
	\item Sicherstellen, dass immer eine der Deadlock-Bedingungen nicht erfüllt ist (Regeln für die Nutzung / erzwungene Freigabe etc).
	\item Zukünftigen Ressourcenbedarf der Prozesse analysieren und Zustände erkennen/verbieten, die zu Deadlocks führen.
	\item Bereits eingetretenden Deadlock erkennen und auflösen.
\end{itemize}


\newpage
\section{Verteilte vs. monolithische Betriebssysteme}
\subsection{Strategien / Varianten}
\begin{itemize}
	\item Hardware / Software / Benutzung
	\item Client-Server-System: Viele Clients greifen auf einen oder mehrere Server bzw. Services zu.
	\item Verteiltes Dateisystem: Ein über mehrere Server verteiltes virtuelles Dateisystem steht Clients zur transparenten Benutzung zur Verfügung.
	\item Netzwerkfähiges Betriebssystem: das Betriebssystem stellt systemübergreifende Funktionalität zur Verfügung.
	\item Verteiltes Betriebssystem: Das Betriebssystem selbst ist verteilt, für Benutzer und Anwendungen ist dies nicht sichtbar.
	\item Verteilte Anwendung: Durch die Programmierung der Anwendung wird das verteilte System erstellt – das Programm muss in der Regel die Verteillogik kennen.
\end{itemize}

\subsection{Vorteile \& Risiken}

\subsubsection{Vorteile}
\begin{itemize}
	\item Verteilung versus Parallelität
	\item Lastverteilung
	\item Leistungssteigerung
	\item Skalierbarkeit/Flexibilität
	\item Sicherheit/Zuverlässigkeit/Verfügbarkeit
	\item Preis/Leistungbzw.Kostenreduktion
	\item Gemeinsame Datennutzung
	\item Gemeinsame Nutzung teurer Peripherie
	\item Applikatorische Verteilung
\end{itemize}

\subsubsection{Risiken}
\begin{itemize}
	\item Erhöhte Komplexität
	\item Erschwerte Fehlersuche	
	\item Verlängerte Abhängigkeitsketten
	\item Sicherheitsdispositiv wird aufwendiger
	\item Nicht alle Subsysteme eignen sich für die Verteilung
	\item Sicherstellung der Konsistenz und Synchronisation
\end{itemize}

\subsection{Anforderungen an verteilte Betriebssysteme}
\begin{itemize}
	\item Gemeinsamer Kernel Code und System Calls
	\item Gemeinsam genutzter Hauptspeicher
	\item Gemeinsames Prozess-Steuersystem
	\item Gemeinsames Dateisystem
	\item Gemeinsame Interprozess-Kommunikation
	\item Gemeinsame Ausnahmebehandlung
	\item Gemeinsame Synchronisationsmechanismen
	\item Gemeinsames System Management
	\item Spezialisierte Funktionen für das Management der Verteilung
	\item Transparente Benutzerschnittstelle
\end{itemize}

\subsubsection{Transparenz}
\begin{description}
	\item[Ortstransparenz:] \hfill \\ Der Ort der genutzten Ressourcen / erbrachten Dienste ist für den Anwender nicht sichtbar.
	\item[Migrationstransparenz:] \hfill \\ Ressourcen können verlagert werden, ohne dass sich ihr Name bzw. ihre Nutzung verändert.
	\item[Replikationstransparenz:] \hfill \\ Anwender können nicht erkennen, wie viele Instanzen es gibt.
	\item[Nebenläufigkeitstransparenz:] \hfill \\ mehrere Anwender können die Ressourcen automatisch gemeinsam und unabhängig voneinander nutzen.
	\item[Parallelitätstransparenz:] \hfill \\ Aktivitäten können parallel bzw. Nebenläufig ausgeführt werden, ohne dass der Anwender es bemerkt.
\end{description}

\subsubsection{Flexibilität}
Services sollen dynamisch an den Bedarf anpassbar sein, entweder durch administrative Eingriffe oder durch Eigenkonfiguration zur Laufzeit. Änderungen sollen keinen kompletten Neustart des verteilten Systems erfordern.

\subsubsection{Zuverlässigkeit}
\begin{itemize}
	\item Erhöhte Zuverlässigkeit gegenüber Einzel- systemen trotz additiver Ausfallwahrschein- lichkeiten
	\item End-zu-End Verfügbarkeit statt Komponentenverfügbarkeit.
	\item Sicherheit – gleiche Richtlinien \& Umsetzung im gesamten verteilten System.
	\item Fehlertolerenz
	\item Automatisches Recovery von Komponenten.
\end{itemize}

\subsubsection{Performance}
\begin{itemize}
	\item Verteilungs- und Kommunikations- Mehraufwand muss den Aufwand wert sein
	\item Wiederholbarkeit / Determinismus von Leistungsindikatoren.
	\item Abhängigkeit von nicht direkt kontrollier- baren Komponenten
	\item End-zu-End Performance statt Komponenten-Performance.
\end{itemize}

\subsubsection{Skalierbarkeit}
Angebotsseite:
\begin{itemize}
	\item Statische oder dynamische Zufügung / Weg- nahme von Servern.
	\item Verrechnung: Durchschnitt oder peaks?
	\item Vermeiden von „Flaschenhälsen“ durch zu starke Serialisierung.
\end{itemize}
Dienstnehmerseite:
\begin{itemize}
	\item Nicht vorhersagbare Anzahl Dienstnehmer.
	\item Einhaltung von Dienstgütegarantien
\end{itemize}

\subsection{Synchronisation in verteilten Systemen}
\begin{description}
	\item[Zeitsynchronisation] \hfill
		\begin{itemize}
			\item Absolut
			\item Relativ
			\item Vorstellen ist einfacher als Rückstellen
		\end{itemize}
	\item[Gegenseitiger Ausschluss über Systemgrenzen] \hfill
		\begin{itemize}
			\item Zentraler Algorithmus (mit Redundanz)
				\begin{itemize}
					\item Zuverlässig, berechenbar
					\item Starke Serialisierung
				\end{itemize}
			\item Verteilter Algorithmus
				\begin{itemize}
					\item Zeitliche Ordnung im Gesamtsystem
					\item Getimte Nachricht an alle Prozesse und Warten auf Bestätigung 
				\end{itemize}
			\item Token Ring Algorithmus
				\begin{itemize}
					\item Explizite Erlaubnis zum Zugriff in vordefinierter Reihenfolge
					\item Suboptimale Ressourcennutzung / Wartezeiten
				\end{itemize}
		\end{itemize}
\end{description}

\newpage
\section{Sicherheitsaspekte im Betriebssystem}
\includegraphics[scale=0.35]{onion_security.png}
\subsection{Typische Schwachstellen}
\begin{itemize}
	\item Lieferwege (physisch, aber vor allem elektronisch)
	\item Software-Voreinstellungen(Defaults)
	\item Funktionale Fehler der Software / Ausnutzbarkeit von Nebeneffekten
	\item Nicht getestete oder nicht autorisierte Änderungen
	\item Fremd-/Fernzugriffe oder Auslagerung aus dem Sicherheits-Kontext
	\item Der Benutzer
	\item Der Administrator
\end{itemize}
\subsection{Ursachen}
\begin{itemize}
	\item Komplexes Zusammenwirken verschiedener Effekte
	\item Ungerechtfertigtes Vertrauen
	\item Gutwilligkeit der Beteiligten
	\item Fehlerhafte Ausführung und/oder Kontrolle
	\item Böswilligkeit / Vorsatz
\end{itemize}
\subsection{Sicherheits-Management}
\includegraphics[scale=0.35]{sicherheits_management.png}
\subsection{Risiko-Management}
\begin{itemize}
	\item Die wissentliche oder unwissentliche Akzeptanz einer Verlustwahrscheinlichkeit und möglichen Schadenhöhe.
	\item Risiko-Management durch:
		\begin{itemize}
			\item Analyse
			\item Vermeidung (nur bedingt möglich, ...)
			\item \"Ubertragung (Versicherung, Werkschutz,...)
			\item Begrenzung (präventive Massnahmen,...)
			\item Akzeptanz (formaler, dokumentierter Willensakt)
			\item Ignoranz (wegschauen,...)
		\end{itemize}
\end{itemize}
Meist gibt es eine Mischung der Massnahmen, abhängig vom Risikotyp, der Risikokultur, den Kostenfolgen usw.
\end{document}

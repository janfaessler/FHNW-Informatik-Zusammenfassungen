\documentclass[a4paper, 11pt]{article}

%Math
\usepackage{amsmath}
\usepackage{amsfonts}
\usepackage{amssymb}
\usepackage{amsthm}
\usepackage{ulem}
\usepackage{stmaryrd} %f\UTF{00FC}r Blitz!

%PageStyle
\usepackage[ngerman]{babel} % deutsche Silbentrennung
\usepackage[ansinew]{inputenc} % wegen deutschen Umlauten
\usepackage{fontenc}
\usepackage{fancyhdr, graphicx} %for header/footer
\usepackage{wasysym}
\usepackage{fullpage}
\usepackage{textcomp}

% Listings
\usepackage{color}
\usepackage{xcolor}
\usepackage{listings}
\usepackage{caption}

% Code listenings
\DeclareCaptionFont{white}{\color{white}}
\DeclareCaptionFormat{listing}{\colorbox{gray}{\parbox{\textwidth}{#1#2#3}}}
\captionsetup[lstlisting]{format=listing,labelfont=white,textfont=white}

\definecolor{DarkPurple}{rgb}{0.4,0.1,0.4}
\definecolor{DarkCyan}{rgb}{0.0,0.5,0.4}
\definecolor{LightLime}{rgb}{0.4,0.6,0.5}
\definecolor{Blue}{rgb}{0.0,0.0,1.0}
 
\lstdefinestyle{JavaStyle}{
 language=Java,
 basicstyle=\footnotesize\ttfamily, % Standardschrift
 numbers=left,               % Ort der Zeilennummern
 numberstyle=\tiny,          % Stil der Zeilennummern
 stepnumber=5,              % Abstand zwischen den Zeilennummern
 numbersep=5pt,              % Abstand der Nummern zum Text
 tabsize=2,                  % Groesse von Tabs
 extendedchars=true,         %
 breaklines=true,            % Zeilen werden Umgebrochen
 frame=b,         
 %commentstyle=\itshape\color{LightLime}, Was isch das? O_o
 %keywordstyle=\bfseries\color{DarkPurple}, und das O_o
 basicstyle=\footnotesize\ttfamily,
 stringstyle=\color[RGB]{42,0,255}\ttfamily, % Farbe der String
 keywordstyle=\color[RGB]{127,0,85}\ttfamily, % Farbe der Keywords
 commentstyle=\color[RGB]{63,127,95}\ttfamily, % Farbe des Kommentars
 showspaces=false,           % Leerzeichen anzeigen ?
 showtabs=false,             % Tabs anzeigen ?
 xleftmargin=17pt,
 framexleftmargin=17pt,
 framexrightmargin=5pt,
 framexbottommargin=4pt,
 showstringspaces=false      % Leerzeichen in Strings anzeigen ?        
}

%Config
\renewcommand{\headrulewidth}{0pt}
\setlength{\headheight}{15.2pt}
\pagestyle{plain}

%Metadata
\title{Entwurfsmuster}
\author{Jan Fässler}
\date{3. Semester (HS 2012)}
\fancyfoot[C]{If you use this documentation for a exam, you should offer a beer to the authors!}

% hier beginnt das Dokument
\begin{document}

% Titelbild
\maketitle
\thispagestyle{fancy}

\newpage

% Inhaltsverzeichnis
\pagenumbering{Roman}
\tableofcontents	  	


\newpage
\setcounter{page}{1}
\pagenumbering{arabic}

% Inhalt Start

\section{Java Collection Framework}
\subsection{Die Interfaces}
\begin{center}
\includegraphics[scale=0.3]{jcf-interfaces.png}
\end{center}

\subsection{Der Iterator}
Ein Iterator ist immer zwischen zwei Elemente. Es gibt also in einer Collection mit n Elementen, n + 1 mögliche Positionen an denen der Iterator stehen kann.
\lstinputlisting[language=java,caption=Iterator,style=JavaStyle]{iterator-interface.java}

\subsection{Die Implementierung}
\includegraphics[scale=0.46]{jcf-implementations.png} 
\begin{description}
	\item[ArrayList] \hfill \\ Eine Implementierung welche ein Array darstellt bei dem man die grösse verändern kann.
	\item[LinkedList] \hfill \\ Eine doppelt verkettete Liste.
	\item[HashSet] \hfill \\ Die Elemente werden in einer Hash-Tabelle gespeichert.
	\item[TreeSet] \hfill \\ Die Elemente werden in einer Baumstruktur gespeichert.
	\item[Maps] \hfill \\ Eine Map ist wie ein Wörterbuch aufgebaut. Jeder Eintrag besteht aus einem Schlüssel (key) und dem zugehörigen Wert (value). Jeder Schlüssel darf in einer Map nur genau einmal vorhanden sein.
\end{description}
Wenn eine Klasse ein Interface implementiert, müssen immer alle Funktionen des Interface implementiert werden. In einer abstrakten Collection-Klasse können alle Funktionen bis auf zwei realisiert werden. Für das Hinzufügen und für den Iterator benötigt es Kentnisse über die Datenstruktur. Hier das ein Beispiel einer Abstrakten Klasse:
\lstinputlisting[language=java,caption=Abstract Collection,style=JavaStyle]{abstract-collection.java}

\section{Design Pattern}
Entwurfsmuster (englisch design patterns) sind bewährte Lösungsschablonen für wiederkehrende Entwurfsprobleme sowohl in der Architektur als auch in der Softwarearchitektur und -entwicklung. Sie stellen damit eine wiederverwendbare Vorlage zur Problemlösung dar, die in einem bestimmten Zusammenhang einsetzbar ist.

\subsection{Types of Patterns}
\begin{description}
	\item[Software Pattern] \hfill
		\begin{itemize}
			\item Architectural Pattern (system design)
			\item Design Pattern (micro architectures)
			\item Coding Pattern / Idioms (low level)
		\end{itemize}
	\item[Analysis Pattern] \hfill
		\begin{itemize}
			\item Recurring \& reusable analysis models used in requirements engineering
		\end{itemize}
	\item[Organizational Patterns] \hfill
		\begin{itemize}
			\item Structure of organizations \& projects: XP, SCRUM
		\end{itemize}
	\item[Domain-Specific patterns] \hfill
		\begin{itemize}
			\item UI patterns, security patterns
		\end{itemize}
	\item[Anti-Patterns] \hfill
		\begin{itemize}
			\item Refactoring
		\end{itemize}
\end{description}

\subsection{Pattern Classification}

\includegraphics[scale=0.5]{pattern-classification.png}

\newpage
\section{Observer Pattern}
Also Known As \textbf{Publish-Subscribe} or \textbf{Listener Pattern}.

\subsection{Intent}
\begin{itemize}
	\item 1:*-Relation between objects which allows to inform the dependent objects about state changes
	\item Consistency assurance between cooperating objects without connecting them too much
	\item Notification of a dependent object without knowing it
\end{itemize}

\subsection{Motivation}
\begin{center}
	\includegraphics[scale=0.4]{observer-motivation.png}
\end{center}

\subsection{Structure}
\begin{center}
	\includegraphics[scale=0.5]{observer-structure.png}
\end{center}

\subsection{Participants}
\begin{description}
	\item[Subject] \hfill
		\begin{itemize}
			\item Knows its observers only through the Observer interface
			\item Offers methods to attach or detach an observer
		\end{itemize}
	\item[Observer] \hfill
		\begin{itemize}
			\item Defines interface to publish notifications
		\end{itemize}
	\item[Concrete Subject] \hfill
		\begin{itemize}
			\item Stores state of interest to concrete observer objects
			\item Notifies its observers when its state changes
		\end{itemize}
	\item[Concrete Observer] \hfill
		\begin{itemize}
			\item Implements Observer interface to keep its state consistent with the subject
			\item May maintain a reference to a concrete subject
		\end{itemize}
\end{description}

\subsection{Collaborations}
\begin{center}
	\includegraphics[scale=0.3]{observer-collaborations.png}
\end{center}

\subsection{Consequences}
\begin{description}
	\item[Decoupling] \hfill
		\begin{itemize}
			\item Subject only knows Observer interface, no concrete observers
			\item Update = dynamically bound => up-calls
			\item Subject and observer may belong to different abstraction layers
		\end{itemize}
	\item[Support for broadcast communication] \hfill
		\begin{itemize}
			\item Notification is broadcast, subject does not care about number of observers
			\item It is up to the observer to handle or ignore notifications
		\end{itemize}
	\item[Unexpected updates] \hfill
		\begin{itemize}
			\item A simple operation on a subject may cause a cascade of updates
		\end{itemize}
\end{description}

\subsection{Beispiel}
\lstinputlisting[language=java,caption=Beispiel Observer Pattern,style=JavaStyle]{observer.java}


% Inhalt Ende
\end{document}
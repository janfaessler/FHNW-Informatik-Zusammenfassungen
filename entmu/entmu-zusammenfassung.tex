\documentclass[a4paper, 11pt]{article}

%Math
\usepackage{amsmath}
\usepackage{amsfonts}
\usepackage{amssymb}
\usepackage{amsthm}
\usepackage{ulem}
\usepackage{stmaryrd} %f\UTF{00FC}r Blitz!

%PageStyle
\usepackage[ngerman]{babel} % deutsche Silbentrennung
\usepackage[ansinew]{inputenc} % wegen deutschen Umlauten
\usepackage{fontenc}
\usepackage{fancyhdr, graphicx} %for header/footer
\usepackage{wasysym}
\usepackage{fullpage}
\usepackage{textcomp}

% Listings
\usepackage{color}
\usepackage{xcolor}
\usepackage{listings}
\usepackage{caption}

% Code listenings
\DeclareCaptionFont{white}{\color{white}}
\DeclareCaptionFormat{listing}{\colorbox{gray}{\parbox{\textwidth}{#1#2#3}}}
\captionsetup[lstlisting]{format=listing,labelfont=white,textfont=white}
 
\lstdefinestyle{JavaStyle}{
 language=Java,
 basicstyle=\footnotesize\ttfamily, % Standardschrift
 numbers=left,               % Ort der Zeilennummern
 numberstyle=\tiny,          % Stil der Zeilennummern
 %stepnumber=2,              % Abstand zwischen den Zeilennummern
 numbersep=5pt,              % Abstand der Nummern zum Text
 tabsize=2,                  % Groesse von Tabs
 extendedchars=true,         %
 breaklines=true,            % Zeilen werden Umgebrochen
 frame=b,         
 stringstyle=\color{white}\ttfamily, % Farbe der String
 showspaces=false,           % Leerzeichen anzeigen ?
 showtabs=false,             % Tabs anzeigen ?
 xleftmargin=17pt,
 framexleftmargin=17pt,
 framexrightmargin=5pt,
 framexbottommargin=4pt,
 showstringspaces=false      % Leerzeichen in Strings anzeigen ?        
}

%Config
\renewcommand{\headrulewidth}{0pt}
\setlength{\headheight}{15.2pt}
\pagestyle{plain}

%Metadata
\title{Entwurfsmuster}
\author{Jan F�ssler}
\date{3. Semester (HS 2012)}
\fancyfoot[C]{If you use this documentation for a exam, you should offer a beer to the authors!}

% hier beginnt das Dokument
\begin{document}

% Titelbild
\maketitle
\thispagestyle{fancy}

\newpage

% Inhaltsverzeichnis
\pagenumbering{Roman}
\tableofcontents	  	


\newpage
\setcounter{page}{1}
\pagenumbering{arabic}

% Inhalt Start

\section{Java Collection Framework}
\subsection{Die Interfaces}
\begin{center}
\includegraphics[scale=0.3]{jcf-interfaces.png}
\end{center}

\subsection{Der Iterator}
Ein Iterator ist immer zwischen zwei Elemente. Es gibt also in einer Collection mit n Elementen, n + 1 m�gliche Positionen an denen der Iterator stehen kann.
\lstinputlisting[language=java,caption=Iterator,style=JavaStyle]{iterator-interface.java}

\subsection{Die Implementierung}
\includegraphics[scale=0.46]{jcf-implementations.png} 
\begin{description}
	\item[ArrayList] \hfill \\ Eine Implementierung welche ein Array darstellt bei dem man die gr�sse ver�ndern kann.
	\item[LinkedList] \hfill \\ Eine doppelt verkettete Liste.
	\item[HashSet] \hfill \\ Die Elemente werden in einer Hash-Tabelle gespeichert.
	\item[TreeSet] \hfill \\ Die Elemente werden in einer Baumstruktur gespeichert.
	\item[Maps] \hfill \\ Eine Map ist wie ein W�rterbuch aufgebaut. Jeder Eintrag besteht aus einem Schl�ssel (key) und dem zugeh�rigen Wert (value). Jeder Schl�ssel darf in einer Map nur genau einmal vorhanden sein.
\end{description}
Wenn eine Klasse ein Interface implementiert, m�ssen immer alle Funktionen des Interface implementiert werden. In einer abstrakten Collection-Klasse k�nnen alle Funktionen bis auf zwei realisiert werden. F�r das Hinzuf�gen und f�r den Iterator ben�tigt es Kentnisse �ber die Datenstruktur. Hier das ein Beispiel einer Abstrakten Klasse:
\lstinputlisting[language=java,caption=Abstract Collection,style=JavaStyle]{abstract-collection.java}

% Inhalt Ende
\end{document}
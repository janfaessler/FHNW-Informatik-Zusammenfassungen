\documentclass[11pt,a4paper]{article}
\usepackage[utf8]{inputenc}
\usepackage{amsmath}
\usepackage{amsfonts}
\usepackage{amssymb}
\author{platzh1rsch}
\title{Diskrete Mathematik 1}
\begin{document}



\section{Mengen und Relationen}
\subsection{Naive Mengenlehre}
- Georg Cantor 1845 -1918 \\
Menge: "Sammlung" von Objekten\\
Diese Objekte heissen Elemente.\\
Notation: X $ /in $ M $\rightarrow$ X ist Element von M\\
Eine Menge ist durch ihre Elemente eindeutig bestimmt.\\
\\
Bsp: M = {1,2,3}, M = N $\rightarrow$ N = {3,1,2}\\
\\
Beschreibung von Mengen\\
1. Durch Aufzählung: M = { 1,2,3 }\\
2. Durch Prädikate:  M = ${x | P(x)}$  "Menge aller x, die das Prädikat P erfüllen"
3. grafische Darstellung (Venn-Diagramme)\\
Bsp. $a \in A, d \in B, c \in A, c \in B$\\

\subsubsection{Notation}
$\forall x \in G: $   "Für alle x aus der Menge G ..."\\
$\exists x \in G: $   "Es existiert ein Element x in der Menge G ..."\\\\
Beispiele:\\
1. G := N = {0,1,2,3...}\\
   A := {1,2}\\
   B := {3,4}\\
$   A B = \emptyset$\\

\subsubsection{Satz 1}
1. G Grundmenge \\
2. A, B, C Teilmengen von G \\

\subsection{weitere Mengen-Konstruktionen}
\subsubsection{Potenzmenge}
\textbf{Definition:} $ P(M) := { x | x M } $ Potenzmenge von M\\
Die Menge aller Teilmengen von M\\
\textbf{Beispiele}\\
a) $ M:={1} \rightarrow P(M) = {\emptyset,{1}}$\\
b) $ M:= {1,2,3} \rightarrow P(M) = {\emptyset, {1},{2},{3},{1,2},{2,3},{1,3},{1,2,3}}$\\
c) $ M:= \emptyset \rightarrow P(M) = {\emptyset}$

\subsubsection{das kartesische Produkt}
Seien A, B Mengen, $a \in A$, $b \in B$\\
\textbf{Definition:} Das Symbol (a,b) heisst das geordnete Paar von a und b.\\
\textbf{Bemerkung:} (a,b) = (c,d) $\rightarrow$ a=c und b=d\\\\

\textbf{Definition:} Seien A,B Mengen\\
$AxB := { (x,y) | x \in A, y \in B }$ heisst das kartesische Produkt von A und B.\\\\

\textbf{Beispiel:} \\
a) ${1,2,3} x {4,5}$ // i.a. $AxB \neq BxA$\\
= { (1,4),(1,5),(2,4),(2,5),(3,4),(3,5) } \\
b) {1,2}x{1,2} = {(1,1),(1,2),(2,1),(2,2)}\\
c) A = {a,b}\\
$Ax\emptyset$ = {(a,$\emptyset$),(b,$\emptyset$)}\\

\subsubsection{Partitionen}
Gegeben eine Menge M\\
\textbf{Definition:} Eine Partition von M ist eine Menge $\pi$\\
$\pi := {Ai | i \in I } $\\
mit \\
1.) $Ai \neq \emptyset$\\
2.) Ai $\subset$ M \\
3.) Ai $\cap$ AJ = $\emptyset$\\
4.) $\cup$ Ai = M = $ A1 \cup A2 \cup A3 ... $

\end{document}
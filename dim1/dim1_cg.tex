\documentclass[11pt,a4paper]{article}
\usepackage[utf8]{inputenc}
\usepackage{amsmath}
\usepackage{amsfonts}
\usepackage{amssymb}
\author{platzh1rsch}
\title{Diskrete Mathematik 1}
\begin{document}



\section{Mengen und Relationen}
\subsection{Naive Mengenlehre}
- Georg Cantor 1845 -1918 \\
Menge: "Sammlung" von Objekten\\
Diese Objekte heissen Elemente.\\
Notation: X $  $ M || X ist Element von M\\
Eine Menge ist durch ihre Elemente eindeutig bestimmt.\\
\\
Bsp: M = {1,2,3}, M = N --> N = {3,1,2}\\
\\
Beschreibung von Mengen\\
1. Durch Aufzählung: M = { 1,2,3 }\\
2. Durch Prädikate:  M = ${x | P(x)}$  "Menge aller x, die das Prädikat P erfüllen"
3. grafische Darstellung (Venn-Diagramme)\\
Bsp. $a \in A, d \in B, c \in A, c \in B$\\

\subsubsection{Notation}
$\forall x \in G: $   "Für alle x aus der Menge G ..."\\
$\exists x \in G: $   "Es existiert ein Element x in der Menge G ..."\\\\
Beispiele:\\
1. G := N = {0,1,2,3...}\\
   A := {1,2}\\
   B := {3,4}\\
$   A B = \emptyset$\\
2. 

\subsubsection{Satz 1}
1. G Grundmenge \\
2. A, B, C Teilmengen von G \\

\subsection{weitere Mengen-Konstruktionen}
\subsubsection{Potenzmenge}
\textbf{Definition:} $ P(M) := { x | x M } $ Potenzmenge von M\\
Die Menge aller Teilmengen von M\\
\textbf{Beispiele}\\
a) $ M:={1} \rightarrow P(M) = {\emptyset,{1}}$\\
b) $ M:= {1,2,3} \rightarrow P(M) = {\emptyset, {1},{2},{3},{1,2},{2,3},{1,3},{1,2,3}}$\\
c) $ M:= \emptyset \rightarrow P(M) = {\emptyset}$

\end{document}
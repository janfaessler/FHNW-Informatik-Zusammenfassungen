\documentclass[a4paper, 11pt]{article}

%Math
\usepackage{amsmath}
\usepackage{amsfonts}
\usepackage{amssymb}
\usepackage{amsthm}
\usepackage{ulem}
\usepackage{stmaryrd} %f\UTF{00FC}r Blitz!

%PageStyle
\usepackage[ngerman]{babel} % deutsche Silbentrennung
\usepackage[ansinew]{inputenc} % wegen deutschen Umlauten
\usepackage{fontenc}
\usepackage{fancyhdr, graphicx} %for header/footer
\usepackage{wasysym}
\usepackage{fullpage}
\usepackage{textcomp}

% Listings
\usepackage{color}
\usepackage{xcolor}
\usepackage{listings}
\usepackage{caption}

% Code listenings
\DeclareCaptionFont{white}{\color{white}}
\DeclareCaptionFormat{listing}{\colorbox{gray}{\parbox{\textwidth}{#1#2#3}}}
\captionsetup[lstlisting]{format=listing,labelfont=white,textfont=white}
 
\lstdefinestyle{JavaStyle}{
 language=Java,
 basicstyle=\footnotesize\ttfamily, % Standardschrift
 numbers=left,               % Ort der Zeilennummern
 numberstyle=\tiny,          % Stil der Zeilennummern
 stepnumber=5,              % Abstand zwischen den Zeilennummern
 numbersep=5pt,              % Abstand der Nummern zum Text
 tabsize=2,                  % Groesse von Tabs
 extendedchars=true,         %
 breaklines=true,            % Zeilen werden Umgebrochen
 frame=b,         
 %commentstyle=\itshape\color{LightLime}, Was isch das? O_o
 %keywordstyle=\bfseries\color{DarkPurple}, und das O_o
 basicstyle=\footnotesize\ttfamily,
 stringstyle=\color[RGB]{42,0,255}\ttfamily, % Farbe der String
 keywordstyle=\color[RGB]{127,0,85}\ttfamily, % Farbe der Keywords
 commentstyle=\color[RGB]{63,127,95}\ttfamily, % Farbe des Kommentars
 showspaces=false,           % Leerzeichen anzeigen ?
 showtabs=false,             % Tabs anzeigen ?
 xleftmargin=17pt,
 framexleftmargin=17pt,
 framexrightmargin=5pt,
 framexbottommargin=4pt,
 showstringspaces=false      % Leerzeichen in Strings anzeigen ?        
}

%Config
\renewcommand{\headrulewidth}{0pt}
\setlength{\headheight}{15.2pt}
\pagestyle{plain}

%Metadata
\title{Algorithmen \& Datenstrukturen 2}
\author{Jan F�ssler}
\date{3. Semester (HS 2012)}
\fancyfoot[C]{If you use this documentation for a exam, you should offer a beer to the authors!}

% hier beginnt das Dokument
\begin{document}

% Titelbild
\maketitle
\thispagestyle{fancy}

\newpage

% Inhaltsverzeichnis
\pagenumbering{Roman}
\tableofcontents	  	


\newpage
\setcounter{page}{1}
\pagenumbering{arabic}

% Inhalt Start

\section{Listen}
Eine verkettete Liste (linked list) ist eine dynamische Datenstruktur zur Speicherung von Objekten. Sie eignen sich f�r das speichern einer unbekannten Anzahl von Objekten, sofern kein direkter Zugriff auf die einzelnen Objekte ben�tigt wird. Jedes Element in einer Liste muss neben den Nutzinformationen auch die notwendigen Referenzen zur Verkettung enthalten.\\ 
Es gibt drei verschiedene Arten von Listen:\\
\begin{center}
\includegraphics[scale=0.4]{listen.png}
\end{center}

\lstinputlisting[language=java,caption=einfache Linked List,style=JavaStyle]{linked_list.java}


\subsection{Stack}
Der Stack ist eine dynamische Datanstruktur bei der man nur auf das oberste Element des Stabels zugreifen (top), ein neues Element auf den Stabel legen (push) oder das oberste Element des Stapels entfernen (pop) kann.
\lstinputlisting[language=java,caption=Implementierung eines Stacks,style=JavaStyle]{stack.java}

\subsection{Erweiterte Liste}
Dies ist mal eine m�gliche und vor allem nur teilweise Implementierung einer doppelt verlinkten Liste. Die Implementierung des Iterators und der sortierung sind ausgeklammert in Unterkapitel.
\lstinputlisting[language=java,caption=Liste mit Iterator,style=JavaStyle]{advanced_list.java}

\subsubsection{Iterators}
Die Schnittstelle java.util.Iterator, erlaubt das Iterieren von Containerklassen. Jeder Iterator stellt Funktionen namens next(), hasNext() sowie eine optionale Funktion namens remove() zur Verf�gung. Der folgende ListIterator stellt auch noch Funktionen f�r r�ckwertsiterieren zur Verf�gung, sowie die M�glichkeit den aktuellen Index abzufragen. Zudem kann damit noch direkt �ber den Iterator Elemente eingef�gt oder ersetzt werden.

\lstinputlisting[language=java,caption=Iterators,style=JavaStyle]{linked_list_iterators.java}

\subsubsection{Merge Sort}
\lstinputlisting[language=java,caption=Merge Sort,style=JavaStyle]{linked_list_mergesort.java}

% Inhalt Ende
\end{document}
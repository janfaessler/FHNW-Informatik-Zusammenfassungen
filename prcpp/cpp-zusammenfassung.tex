\documentclass[a4paper, 11pt]{article}

%Math
\usepackage{amsmath}
\usepackage{amsfonts}
\usepackage{amssymb}
\usepackage{amsthm}
\usepackage{ulem}
\usepackage{stmaryrd} %f\UTF{00FC}r Blitz!

%PageStyle
\usepackage[ngerman]{babel} % deutsche Silbentrennung
\usepackage[ansinew]{inputenc} % wegen deutschen Umlauten
\usepackage{fontenc}
\usepackage{fancyhdr, graphicx} %for header/footer
\usepackage{wasysym}
\usepackage{fullpage}
\usepackage{textcomp}

% Listings
\usepackage{color}
\usepackage{xcolor}
\usepackage{listings}
\usepackage{caption}

% Code listenings
\DeclareCaptionFont{white}{\color{white}}
\DeclareCaptionFormat{listing}{\colorbox{gray}{\parbox{\textwidth}{#1#2#3}}}
\captionsetup[lstlisting]{format=listing,labelfont=white,textfont=white}
 
\lstdefinestyle{JavaStyle}{
 language=Java,
 basicstyle=\footnotesize\ttfamily, % Standardschrift
 numbers=left,               % Ort der Zeilennummern
 numberstyle=\tiny,          % Stil der Zeilennummern
 stepnumber=5,              % Abstand zwischen den Zeilennummern
 numbersep=5pt,              % Abstand der Nummern zum Text
 tabsize=2,                  % Groesse von Tabs
 extendedchars=true,         %
 breaklines=true,            % Zeilen werden Umgebrochen
 frame=b,         
 %commentstyle=\itshape\color{LightLime}, Was isch das? O_o
 %keywordstyle=\bfseries\color{DarkPurple}, und das O_o
 basicstyle=\footnotesize\ttfamily,
 stringstyle=\color[RGB]{42,0,255}\ttfamily, % Farbe der String
 keywordstyle=\color[RGB]{127,0,85}\ttfamily, % Farbe der Keywords
 commentstyle=\color[RGB]{63,127,95}\ttfamily, % Farbe des Kommentars
 showspaces=false,           % Leerzeichen anzeigen ?
 showtabs=false,             % Tabs anzeigen ?
 xleftmargin=17pt,
 framexleftmargin=17pt,
 framexrightmargin=5pt,
 framexbottommargin=4pt,
 showstringspaces=false      % Leerzeichen in Strings anzeigen ?        
}

%Config
\renewcommand{\headrulewidth}{0pt}
\setlength{\headheight}{15.2pt}
\pagestyle{plain}

%Metadata
\title{C++}
\author{Jan F�ssler}
\date{3. Semester (HS 2012)}
\fancyfoot[C]{If you use this documentation for a exam, you should offer a beer to the authors!}

% hier beginnt das Dokument
\begin{document}

% Titelbild
\maketitle
\thispagestyle{fancy}

\newpage

% Inhaltsverzeichnis
\pagenumbering{Roman}
\tableofcontents	  	


\newpage
\setcounter{page}{1}
\pagenumbering{arabic}

% Inhalt Start

\section{Einleitung}

\subsection{C++ verglichen mit Java}
\subsubsection{Gemeinsamkeiten}
\begin{itemize}
	\item typisierte, objektorientierte Sprache
	\item sehr �hnliche Syntax (Java-Syntax wurde an C++ angelehnt)
	\item �hnliche Grundtypen, Operatoren und Klassenkonzept
\end{itemize}
\subsubsection{Unterschiede}
\begin{itemize}
	\item C++-Programm muss nicht objektorientiert sein
	\item plattformabh�ngiger Maschinencode anstatt Bytecode f�r die VM
	\item C++-Programme ko?nnen aufs unterliegende System zugreifen
	\item Flexibleres Speichermanagement
	\item Flexiblerer Polymorphismus
	\item Effizienz vor Sicherheit
	\item Unterscheidung zwischen Referenzen und Zeigern
	\item keine strikt geschachtelten Namensr�ume
	\item Trennung zwischen Schnittstelle und Implementierung
\end{itemize}

\subsection{C++-Dateien}
\begin{description}
	\item[*.c] \hfill \\
		Dateien, die mit dem C-Compiler kompiliert werden
	\item[*.cpp] \hfill \\
		Dateien, die mit dem C++-Compiler kompiliert werden
	\item[Header-Datei (*.h)] \hfill \\
		enth�lt oft mehrfach ben�tigte Definitionen und wird nicht direkt kompiliert, sondern in eine oder mehrere cpp-Dateien importiert
	\item[*.hpp] \hfill \\
		urspr�nglich als reine C++-Header-Dateien gedacht, werden aber selten verwendet
\end{description}

\subsection{Programmerzeugung}
\subsubsection{Pr�prozessor}
\begin{itemize}
	\item Programmcode darf Makros enthalten
	\item Makros werden unmittelbar vor der Kompilation evaluiert
	\item Bsp. Substitution von Konstanten, bedingte Kompilation
	\item Verwendung: 
		\begin{description}
			\item[\#define] \hfill \\ 
				definiert ein Symbol/Makro mit oder ohne Parameter
			\item[\#undef] \hfill \\ 
				l�scht eine Definition eiens Symbols/Makros, bzw ist danach nicht mehr definiert
			\item[\#ifdef und \#endif] \hfill \\ 
				bedingte Kompilation: die Kompilation eines Textblocks ist abha?ngig von der Definition eines Symbols
		\end{description}
\end{itemize}
\subsubsection{Compiler}
\begin{itemize}
	\item Syntax�berpr�fung des Quellcodes
	\item Erzeugung von Objektdateien (Maschinencode mit unaufgel�sten Verkn�pfungen zu anderen Objektdateien)
	\item Der C++-Compiler ist ein One-Pass-Compiler. Das bedeuted bevor ein Bezeichner (Variable, Klasse usw.) verwendet werden darf, muss er deklariert bzw. definiert werden. Die Deklaration bzw. Definition eines Bezeichners muss vor seiner Benutzung kompiliert werden.
\end{itemize}
\subsubsection{Linker (Binder)}
\begin{itemize}
	\item Erzeugung von Bibliotheken oder ausf�hrbaren Programmen aus einzelnen Objektdateien
	\item Verkn�pfungen zwischen Objektdateien werden aufgel�st
	\item Optimierungen (z.B. Entfernung nicht verwendeter Prozeduren) sind m�glich
\end{itemize}

\newpage
\section{Variablen und Methoden}
\subsection{Global}
\begin{description}
	\item[Java] \hfill
		\begin{itemize}
			\item Main-Methode ist eine Klassenmethode, also Teil einer Klasse
			\item Klassen sind Teil eines Packages (evtl. unbenanntes Package)
			\item Klassenvariablen geho?ren zum Namensraum einer Klasse
		\end{itemize}
	\item[C++] \hfill
		\begin{itemize}
			\item Main-Funktion ist global (Teil des globalen Namensraums)
			\item Klassen, Methoden, Variablen sind Teil eines Namensraums (benannt oder global)
			\item uneingeschr�nkte Sichtbarkeit: aus allen Programmteilen ko?nnen sie verwendet werden (Sichtbarkeit u?ber die Objektdateigrenze hinweg)
			\item Verwendung: wenn immer m�glich vermeiden, da das Information Hiding Prinzip stark unterwandert wird
		\end{itemize}
\end{description}

\subsection{Modular}
\begin{description}
	\item[Sichtbarkeitsbereich] \hfill \\
		... ist beschr�nkt auf die Objektdatei, jedoch �ber Methoden- und Klassengrenzen hinweg
	\item[Einsatzgebiet] \hfill \\
		bei nicht-objektorientierter Programmierung als Ersatz von Klassenvariablen und -methoden (in OO: Einsatz vermeiden)
\end{description} 

\subsection{Einfache Datentypen}
Speicherbedarf der einfachen Datentypen ist Compiler spezifisch. Alle ganzzahligen Datentypen (inkl. char) gibt es vorzeichenlos (unsigned) und vorzeichenbehaftet (signed) in der Zweierkomplementdarstellung.

\subsection{Schl�sselw�rter}
\begin{description}
	\item[typedef] \hfill \\
		Es dient der Festlegung eigener Typenbezeichner.\\ 
		Bsp.: \\ typedef int INT32; \\ typedef unsigned long long int UINT64;
	\item[using (C++11)] \hfill \\
		Dieses kann auch fu?r eigene Typenbezeichner verwendet werden. \\ 
		Bsp.: \\ using INT32 = int; \\ using UINT64 = unsigned long long int;
	\item[auto (C++11)] \hfill \\
		Bei Variablendefinitionen, wo aus dem Initialisierungswert der Variable der Typ der Variable fu?r den Compiler automatisch ersichtlich ist, kann das Schlu?sselwort auto anstatt des konkreten Typs hingeschrieben werden. \\ 
		Bsp.: \\ auto x = 7; \\ double f(); \\ auto g = f();
	\item[decltype (C++11)] \hfill \\
		decltype(x) ist eine Funktion, welche den Deklarationstyp des Ausdruckes x zuru?ckgibt. \\
		Bsp.: \\ decltype(8) y = 8; \\ decltype(g) h = 5.5;
	\item[const] \hfill \\
		Im Gegensatz zu Java wo dieses Wort reserviert aber nicht verwendet wird, hat es in C++ vielf�ltiger Einsatz mit unterschiedlicher Semantik. Die hier verwendete Semantik: nach Initialisierung nur noch lesender Zugriff. \\
		Beispiele: \\
		const unsigned int SIZE = 1000; \\ const auto LENGTH = 500; \\ const char GRADES[] = { 'A', 'B', 'C', 'D', 'E', 'F' };  \\ const char NOTEN[] = { 1, 2, 3, 4, 5, 6 }; \\ double const PI = 3.141596; \\ auto const PID2 = PI/2;
	\item[constexpr (C++11)] \hfill \\
	Verallgemeinerung des Schl�sselwort const fu?r konstante Ausdr�cke, welche auch Funktionsaufrufe und als Spezialfall auch Konstruktoren enthalten d�rfen und stellt statische Initialisierung zur Kompilationszeit sicher. \\
	Beispiel: \\ constexpr int getFive() { return 2 + 3; } \\ int array[getFive() + 7];
\end{description}

\subsection{Initialisierungslisten}
\begin{lstlisting}[language=Java, label=Initialisierungsliste, caption=Initialisierungsliste, style=JavaStyle]
#include <initializer_list> struct Tuple {
int value[];
Tuple(initializer_list<int> v);
Tuple(int a, int b, int c); Tuple(initializer_list<int>v,size_tcap); //#3
};
Tuple t1{1, 2, 3};    // Konstruktor #1 wird verwendet
Tuple t2{2, 4, 6, 8}; // Konstruktor #1 wird verwendet
Tuple t3(4, 5, 6};    // Konstruktor #2 wird verwendet
\end{lstlisting}
Die Initialisierungslisten sind ein neuer C++ Typ. Wenn die Initialisierungsliste der einzige Parameter ist, kann wie oben gezeigt vorgegangen werden. Wenn noch weitere Parameter vorhanden sind, dann m�ssen die geschweiften Klammern verschachtelt werden Tuple: \\ t4 = \{\{1, 2, 3, 4\}, 4\};

\newpage
\section{Zeiger \& Referenzen}

\subsection{Zeiger und Adressoperator}
Ein Zeiger zeigt auf eine Speicherstelle des (virtuellen) Adressraums. Zeigen k�nnen im Quellcode Typinformationen mitf�hren. Von jeder Variable und jedem Objekt kann mit dem Adressoperator \& zur Laufzeit die Adresse (Speicherstelle) abgefragt werden. \\ \\
Zeigervariablen bzw. Zeiger zeigen auf g�ltige Speicheradressen (zB. dynamische Objekte, aufs erste Element von Arrays oder auf statische Variablen und Objekte). Sie k�nnen aber auch auf ung�ltige Speicheradressen zeigen. Sie  haben einen Typ \"{}Zeiger auf ...\"{}. Soll eine Zeigervariable auf eine Instanz der Klasse C zeigen, so muss der Typ der Zeigervariablen zur Klasse C zuweisungskompatibel sein
\begin{lstlisting}[language=Java, caption=Zeigerbeispiele, style=JavaStyle]
typedef unsigned int * PUInt32;
char text[] = "test";
unsigned int i = 2;
char c = text[i + 1];
char *p = text, *q = text + 1, *r = &text[i], *s = &c, *t = nullptr, *u = 0;
PUInt32 x = &i;
void *y = x;
\end{lstlisting}

\newpage
\section{Klassen}
\subsection{Deklaration}
In C++ kann man Klassen auf zwei Arten erzeugen:
\begin{description}
	\item[struct] \hfill \\
		in C: Verbund (Record) von verschiedenen Datenfeldern \\
		in C++: �ffentliche Klasse (alle Members sind public per Default)
		\begin{lstlisting}[language=Java, caption=struct-Klasse, style=JavaStyle]
struct Point {
	int m_x, m_y;
	void setY(int y) { m_y = y; }
};
		\end{lstlisting}
	\item[class] \hfill \\
		nur in C++: alle Members sind private per Default
		\begin{lstlisting}[language=Java, caption=class-Klasse, style=JavaStyle]
class Person {
		char m_name[20];
		int m_alter; 
	public:
		char * getName() { return m_name; }
};
		\end{lstlisting}
\end{description}

\subsection{Instanzen}
\begin{lstlisting}[language=Java, caption=Erzeugen von Instanzen, style=JavaStyle]
Point pnt1;                   // auf Stack
Point *pnt2 = new Point();    // auf Heap
Person pers1;                 // auf Stack
Person *pers2 = new Person(); // auf Heap
Person& refP = pers1;
\end{lstlisting}

\begin{lstlisting}[language=Java, caption=Zugriff auf Instanzvariablen und Instanzmethoden, style=JavaStyle]
pnt1.m_x = 3; 
pnt1.setY(pnt1.m_x);
pnt2->m_x = 4; 
pnt2->setY(pnt2->m_x); 
char *name = pers1.getName();
name = pers2->getName();
name = refP.getName();
\end{lstlisting}
% Inhalt Ende
\end{document}